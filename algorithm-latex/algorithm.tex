\documentclass[lang=cn,11pt]{elegantbook}
%加水印
\usepackage{eso-pic}
\newcommand{\watermark}[3]{\AddToShipoutPictureBG{
		\parbox[b][\paperheight]{\paperwidth}{
			\vfill%
			\centering%
			\tikz[remember picture, overlay]%
			\node [rotate = #1, scale = #2] at (current page.center)%
			{\textcolor{gray!80!cyan!30}{#3}};
			\vfill}}}
\usepackage{blindtext}
% title info
\title{算法笔记}
\subtitle{algorithm note}
% bio info
\author{LHesperus}
\institute{UESTC}
\date{2019.09.29-\today}
% extra info
\version{1.00}
\extrainfo{ e}
\logo{logo.png}
\cover{cover.jpg}

\begin{document}
	%% 水印
	\watermark{0}{5}{github:LHesperus}
	\clearpage
	
	\maketitle
	\tableofcontents
	\mainmatter
	\hypersetup{pageanchor=true}
	% add preface chapter here if needed
	\chapter{数论算法}
	\section{最大公约数}
	两个不同时为0 的数a 与b 的\textbf{最大公因数}是同时整除它们的两个最大
	的数,记为gcd(a, b),同时,如果gcd(a,b) = 1,我们称它们\textbf{互素}。
	
$\textbf{O(log(N))}$	\textbf{欧几里得算法/辗转相除法:}$gcd(m,n) = gcd(n,m \mod n)$

递归不会栈溢出:gcd函数的递归层数不会超过$4.785lgN+1.6723,N=\max{(a,b)}$,让gcd递归最多层的是$gcd(F_{n},F_{n-1})$,$F_{n}$是Fibonacci数。
	\section{扩展欧几里得算法}
\textbf{应用:}
	\begin{itemize}
	\item 求解不定方程(如$99x+78b=6$的整数解);
	
	\item 求解模线性方程(线性同余方程);
	
	\item 求解模的逆元;
	\end{itemize}

	\subsection{裴蜀定理}
	若a 和b 是整数,方程ax+by=d 有整数解当且􂎑当gcd(a,b) | d。
	例如,方程3x+6y=2 就不存在整数解,方程3x+6y=3 存在(无数多
	个)整数解,其中一个是x=1,y=0。
	
	这个定理给我们了一个判定形如ax+by=d 的方程是否有整数解的方
	法,但是它并没有告诉我们如何求解。求解这样的方程是扩展欧几里得算法
	的内容。
	
	\subsection{同余}
	$a=b(\mod p)$:$a,b$模$p$后的余数相同。\\
	若\[
	\begin{cases}
	a1=b1(\mod p)\\
	a2=b2(\mod p)\\
	\end{cases}
	\]
	则:	\[
	\begin{cases}
	a1\pm a2=b1 \pm b2 (\mod p)\\
	a1\cdot a2=b1 \cdot b2 (\mod p)\\
	\end{cases}
	\]

	\subsection{乘法逆元}
	如果$ax=1 (\mod p)$,且$gcd(a,p)=1$ (a与p互质),则称a关于模p的乘法逆元为x。
	
	在同余意义下,加减法和乘法都和普通的运算没什么区别,但是唯独
	“除法”有一些区别:\textbf{当没有逆元时无法进行“除法”}!
	
	如$15\times 2 = 20\times 2 (\mod 10)$
	
	但	$15\neq 20 (\mod 10)$
	因为在模10 意义下,2 是没有乘法逆元的。
	
	\subsection{求解模线性方程}
		\begin{example}
		求关于x 的同余方程$ax =1 (\mod b)$ 的最小正整数解。
		其中$0 \leq a, b \leq 2<10^9$,并且保证该方程有解。
		
		如果上述同余方程被满足的话,一定存在整数y 使得$ax = 1+by$,这
		样我们可以直接利用扩展欧几里得算法得出一个解。至于最小正整数解也是
		可以很容易就计算得出,因为在$1 \leq x \leq b$ 中这个方程有唯一解。
	\end{example}
	\section{快速幂}
	\bibliography{reference}

\end{document}